\section {La structure thread}

Un type \verb'thread_t' devait être implémenté pour correspondre à
l'exemple, nous avons décidé qu'il représenterait un pointeur vers une
structure thread que nous allons maintenant détailler.\\

\begin{verbatim}
struct thread {
    ucontext_t uc;
    void *retval;
    struct GList * sleeping_list;

    int basic_prio;
    int current_prio;
    treat_func treat_tab[NB_SIG];
    GList* sig_list;

    /*valgrind stackid*/
    int stackid;
}
\end{verbatim}

Le champ \verb'uc' sert à conserver le contexte d'exécution du thread,
\verb'retval' quant à lui permet de stocker la valeur de retour d'un
thread quand celui-ci est dans la liste "zombie". Mais quand il est
dans la liste "ready", \verb'retval' peut contenir la valeur de retour
du thread qu'il attendait. Le troisième champ est la liste des threads
dormants dans l'attente de la fin de l'exécution de ce thread. 

Les champs \verb$basic_prio$ et \verb$current_prio$ servent à gérer la
priorité des threads.  \verb$basic_prio$ est la priorité que l'on
souhaite donner au thread; \verb$current_prio$ sert à ranger les
threads dans la liste "ready" selon leur priorité. Elle a au départ la
même valeur que \verb$basic_prio$, mais sa valeur peut diminuer (donc
sa priorité augmente), afin d'éviter qu'un thread de priorité faible
n'attende trop longtemps.

Le tableau \verb$treat_tab$ contient les fonctions de traitement
correspondant à chaque type de signal, la liste \verb$sig_list$ est la
liste de ces signaux.

Le champ \verb$stackid$ sert à aider valgrind à trouver la pile du thread. 

\section {Les threads dans l'état "ready" et "zombie"}

Nous avons choisi de stocker ces threads dans deux listes distinctes
pour chacun des états. Les threads "ready" sont ceux prêts à
s'exécuter, et les threads "zombie" ont terminé leur exécution et
attendent que leur valeur de retour soit récupérée.\\ Nous avons pris
comme convention que la tête de la liste "ready" soit le thread en
cours d'exécution.

\section {Utilisation de la GLib}

Pour nous abstraire de l'implémentation du TAD list nous avons utilisé
la GLib, nous fournissant les fonctions nécessaires à la manipulation
des listes de threads tout en nous assurant qu'il n'y aura aucune
fuite mémoire.

Cependant, une autre conséquence de ce choix est l'impossibilité
d'accéder au dernier élément de la liste en temps constant.

\section {L'implémentation des fonctions}

\begin{verbatim}
thread_t thread_self(void);
\end{verbatim}
Avec notre choix de mettre le thread courant en tête de la liste "ready",
cette fonction renvoie simplement la tête de la liste "ready".
~~\\
\begin{verbatim}
int thread_create(thread_t *newthread, void *(*func)(void *), void *funcarg);
\end{verbatim}


Cette fonction a la responsablité d'ajouter le thread main dans la
liste "ready", si c'est la première fois qu'on crée un thread.  Comme
la liste "ready" est initialement vide, la fonction vérifie si cette
liste est vide, et si c'est le cas, elle alloue une instance de la
structure thread, sauvegarde le contexte du main dans cette instance
et la place à la tête de la liste.  Puis elle crée une instance de la
structure thread avec la fonction passée en paramètre et retourne un
thread\_t pointant sur cette structure.  ~~\\
\begin{verbatim}
int thread_yield(void);
\end{verbatim}
Cette fonction fait un $swapcontext$ sur le contexte de la tête de la
liste "ready" (qui est le thread courant) et celui de l'élément
suivant.  ~~\\
\begin{verbatim}
int thread_join(thread_t thread, void **retval);
\end{verbatim}
Cette fonction vérifie tout d'abord si le thread qu'on veut attendre
est dans la liste "ready". Si c'est le cas alors elle se met dans la
liste des dormants de ce thread et passe la main au suivant.  Sinon
elle vérifie si ce thread est dans la liste "zombie", auquel cas
elle prend la valeur de retour de ce thread et continue son exécution.
Enfin si ce thread n'est pas dans la liste "zombie" alors elle
retourne -1 pour dire que l'appel a échoué et que ce thread n'existe
pas.  ~~\\
\begin{verbatim}
void thread_exit(void *retval);
\end{verbatim}
Cette fonction réveille tous les threads de sa liste des dormants et
met à jour son champ retval avec la valeur en paramètre. Puis elle se
met dans la liste "zombie" et passe la main au suivant.

\section{libération de la mémoire}
On suppose qu'un thread est attendu par au moins un autre thread.

Un thread est libéré lorsqu'il est dans la liste "zombie", soit :
\begin{itemize}
\item Lors d'un join d'un autre thread sur ce thread.
\item Lors de la reprise d'un thread qui l'attendait.


Comme nos listes sont fournies par la glib, il est nécessaire
d'utiliser la commande suivante pour utiliser valgrind :
\begin{verbatim}
G_SLICE=always-malloc valgrind --leak-check=full --show-reachable=yes ./example 
\end{verbatim}
Sans cette commande, il considère certaines pertes mémoire possibles
alors qu'elles n'ont pas lieu d'être.
