\section{Préemption}
Nous avons mis en place un préemption pseudo-coopérative, cachée dans
certaines fonctions de notre bibliothèque, de telle sorte que, au bout
d'une certaine durée prédéfinie, le thread en cours d'exécution reçoit
un signal \verb$SIGVTALRM$ et laisse la main au thread suivant dans la
liste "ready".

Cette forme de préemption est basée sur l'utilisation d'un timer, par
l'intermédiaire des fonctions \verb$setitimer$ et
\verb$getitimer$. Nous utilisons un timer virtuel, qui décroit
uniquement lorsque le processus s'exécute, et émet un signal
\verb$SIGVTALRM$ à l'expiration du délai.  La durée du timer est
définie par la variable globale \verb$TIMESLICE$.

Le timer est créé lors de la création du premier thread du programme,
et est désactivé lors de la disparition de ce même thread. Il est
réinitialisé dans \verb$thread_yield$, \verb$thread_join$ et
\verb$thread_exit$, pour que le thread qui reprend la main ait à son
tour le temps d'exécution disponible correspondant à la valeur de
\verb$TIMESLICE$. 

Dans notre version, nous avons donné à la variable \verb$TIMESLICE$
une valeur fixée, mais nous pourrions aller plus loin en lui donner
une valeur variable en fonction de la priorité du thread qui doit
s'exécuter.
